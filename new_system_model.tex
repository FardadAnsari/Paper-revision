\documentclass[10pt]{article}
\usepackage{amsmath, amssymb}
\usepackage{graphicx}

\title{Your Paper Title}
\author{Your Name}
\date{\today}

\begin{document}

\maketitle

\section{System Model}
\label{sec:system_model}

\subsection{System Configuration}
Like most scenarios adopts a single user massive MIMO system in a single cell, in which the base station equips $N_{t} > 1$ transmitting antennas as well as the UE with a single receiving antennas. An orthogonal frequency division multiplexing (OFDM) system with $N_{c}$ sub-carriers for FDD downlink massive MIMO is examined. In the OFDM-MIMO system considered in this work, the subcarriers are assumed to maintain orthogonality.

\subsection{Channel Model}
The time-domain channel impulse response between the $i$-th transmit antenna and UE follows Rayleigh fading:
\[
h_i(\tau) = \sum_{l=1}^L \alpha_{i,l} \delta(\tau-\tau_l), \quad \alpha_{i,l} \sim \mathcal{CN}(0,\sigma_l^2)
\]
where $\alpha_{i,l}$ is the complex gain and $\tau_l$ is the delay of the $l$-th path. The frequency-domain channel matrix for subcarrier $n$ is:
\[
\mathbf{H}_n = \left[H_{1,n}, H_{2,n}, \dots, H_{N_t,n}\right] \in \mathbb{C}^{1 \times N_t}
\]
with elements derived from:
\[
H_{i,n} = \sum_{l=1}^L \alpha_{i,l} e^{-j2\pi n \tau_l/N_c}
\]

\subsection{Pilot Transmission}
We will assume base station sends orthogonal pilots for each subcarrier to the UE $n=1,...,N_{c}$: let pilot matrix $\mathbf{X}_{p,n}\in \mathbb{C}^{N_{t}\times T_{p}}$ be orthogonal ($T_{p} \geq N_{t}$). We will assume on $n^{th}$ sub-carriers the received signal at UE will be equal to: 
\[
\mathbf{Y}_{p,n}=\mathbf{H}_{n}\mathbf{X}_{p,n}+\mathbf{Z}_{n} 
\]
where $\mathbf{Y}_{p,n} \in \mathbb{C}^{1\times T_{p}}$ ($N_{r}=1$) is received pilot at $n^{th}$ subcarrier, $\mathbf{H}_{n} \in \mathbb{C}^{1\times N_{t}}$ is the MIMO channel matrix for subcarrier $n$ and $\mathbf{Z}_{n}\in \mathbb{C}^{1\times T_{p}}$ is AWGN noise with variance $\sigma^{2}$.

\subsection{Channel Estimation}
Let's assume UE uses Least Squares (LS) to estimate $\mathbf{H}_{n}$:
\[
\hat{\mathbf{H}}_{n}=\mathbf{Y}_{p,n}\mathbf{X}^{\dag}_{p,n}
\]
where $\hat{\mathbf{H}}_{n} \in \mathbb{C}^{1\times N_{t}}$ and $\mathbf{X}_{p,n}^{\dag}= \mathbf{X}^{H}_{p,n}(\mathbf{X}_{p,n}\mathbf{X}^{H}_{p,n})^{-1}$. The pilot matrix $\mathbf{X}_{p,n}$ satisfies $\mathbf{X}_{p,n} \mathbf{X}_{p,n}^H = \mathbf{I}_{N_t}$ for $T_p \geq N_t$.

\subsection{CSI Feedback}
For each subcarrier estimated CSI matrices is $\{ \hat{\mathbf{H}}_{1},...,\hat{\mathbf{H}}_{N_{c}} \}$. For simplicity we stack and reshape the channel matrix into:
\[
H_{\text{freq}}=\left[\hat{\mathbf{H}}_{1}^{H},\hat{\mathbf{H}}_{2}^{H},...,\hat{\mathbf{H}}_{N_{c}}^{H}\right]^{H}_{N_{c}\times N_{t}}
\]
$H_{\text{freq}} \in \mathbb{C}^{N_c \times N_t}$ is the stacked CSI matrix across subcarriers. Evidently, $H_{\text{freq}} \in \mathbb{C}^{N_{c}\times N_{t}}$ has $2\times N_{c}\times N_{t}$ float numbers, which is too big for direct feedback in a massive MIMO system and this feedback is required for creating the precoding to be built by base station.

$H_{\text{freq}}$ is the spatial-frequency domain equivalent of the CSI. We use the 2D discrete Fourier transform (DFT) and transfer $H_{\text{freq}}$ into the angle-delay domain as follows to extract the CSI features:
\[
\tilde{\mathbf{H}} = \mathbf{F}_d H_{\text{freq}} \mathbf{F}_a^H
\]
where $\mathbf{F}_d \in \mathbb{C}^{N_{c}\times N_{c}}$ is delay DFT and $\mathbf{F}_a \in \mathbb{C}^{N_{t}\times N_{t}}$ is angular DFT and both of them are unitary DFT matrices.

CSI shows sparsity in the delay domain, with $\tilde{\mathbf{H}}$ having significant values only in the first $N_{i}$ rows because the time of arrival (TOA) between multipaths is limited in duration. Like aforementioned researches, we select first $N_{i}$ rows of $\tilde{\mathbf{H}}$ to create a new channel matrix $\tilde{\mathbf{H}}_f$ as
\[
\tilde{\mathbf{H}}_f = \left[ \tilde{\mathbf{H}} \right]_{1:N_{i}} 
\]

$\tilde{\mathbf{H}}_f$ as is still too heavy for feedback, though, because in a massive MIMO system, $N_{t}$ is a big number. Our goal is to further compress matrix $\tilde{\mathbf{H}}_f$ in order to minimize the weight of the feedback.

\subsection{Deep Learning Compression}
In an effort to lower transmission overheads, DL-based algorithms have recently been used in CSI feedback. An encoder at UE first compresses the CSI data into a codeword:
\[
s = f_{\text{enc}}(\tilde{\mathbf{H}}_f) 
\]
and after that, a feedback channel is used to send the codeword to the BS. A decoder at the BS can recreate the CSI as:
\[
\hat{\tilde{\mathbf{H}}}_f = f_{\text{dec}}(s) 
\]

The compression ratio can be obtained as:
\[
\gamma = \frac{N_{s}}{2\times N_{i}\times N_{t}}  
\]
where $\gamma$, $N_{s}$, $N_{i}$ and $N_{t}$ are compression ratio, codeword length, selected rows from $\tilde{H}$ and number of transmit antenna.

\end{document}